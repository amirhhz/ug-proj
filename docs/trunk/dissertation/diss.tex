%%%%%%%%%%%%%%%%%%%%%%%%%%%%%%%%%%%%%%%%%%%%%%%%%%%%%%%%%%%%%%%%%%%%%%%%%%%%%%%
% PART II PROJECT DISSERTATION - RECOMMENDER SYSTEMS
%
% Amir H. Hajizamani, ahh29@cam.ac.uk
%
%%%%%%%%%%%%%%%%%%%%%%%%%%%%%%%%%%%%%%%%%%%%%%%%%%%%%%%%%%%%%%%%%%%%%%%%%%%%%%%

\documentclass[a4paper,12pt,twoside,notitlepage,draft]{report}

\author{Amir H. Hajizamani}
\title{CST Part II Individual Project Dissertation - Recommender Systems for 
Social Networks}

\usepackage[T1]{fontenc}
\usepackage{a4wide}
\usepackage{parskip}
\usepackage{hyperref}
\usepackage{url}
\usepackage{amsmath}
%\usepackage{proof}
\usepackage{comment}

% Conditionals (for draft mode)
\usepackage{ifthen}

% Headers
\usepackage{fancyhdr}

% TOC customisation
\usepackage{tocloft}

% Captions
\usepackage{subfig}
\usepackage{caption}

% Figures
\usepackage{wrapfig}
\usepackage{graphicx}
\usepackage{tikz}
\usetikzlibrary{positioning,mindmap,chains,shapes.misc,arrows,fit}
\usetikzlibrary{decorations.pathmorphing}

% Tables
\usepackage{booktabs}
\usepackage{colortbl}
\usepackage{longtable}

% Palatino font - remove to go back to normal CMR
\usepackage{mathpazo}

% Colour
\usepackage{color}

% Listings (for code snippets ..._
\usepackage{listings}

% Algorithms
\usepackage[boxed]{algorithm}

%\makeindex

\frenchspacing

%%%%%%%%%%%%%%%%%%%%%%%%%%%%%%%%%%%%%%%%%%%%%%%%%%%%%%%%%%%%%%%%%%%%%%%%%%%%%%%
% The BIG DRAFT SWITCH - Set to true to enable draft mode
%
% !!! This will produce extra output that 
%     is not part of the final dissertation !!!

\newboolean{draft}
\setboolean{draft}{false}

%%%%%%%%%%%%%%%%%%%%%%%%%%%%%%%%%%%%%%%%%%%%%%%%%%%%%%%%%%%%%%%%%%%%%%%%%%%%%%%

%%%%%%%%%%%%%%%%%%%%%%%%%%%%%%%%%%%%%%%%%%%%%%%%%%%%%%%%%%%%%%%%%%%%%%%%%%%%%%%
% Definitions

\def\projtitle{Recommender Systems for Social Networks}
\def\authorname{Amir~H.~Hajizamani} 
\def\authoremail{\url{ahh29@cam.ac.uk}}
\def\authorcollege{St.~John's~College}


%%%%%%%%%%%%%%%%%%%%%%%%%%%%%%%%%%%%%%%%%%%%%%%%%%%%%%%%%%%%%%%%%%%%%%%%%%%%%%%
% Custom Environments

\newcommand{\fixme}[1]{\ifdraft{\color{red} {\bf Fixme: \sc #1}}\fi}
\newcommand{\todo}[1]{\ifdraft{\textsf{\color{red} TODO: #1}}\fi}
\newcommand{\remark}[1]{\ifdraft{\textsf{\color{blue} #1}}\fi}

\newcommand{\Matlab}{\textsc{Matlab} }

%% Colours
\definecolor{darkgreen}{rgb}{0,0.3137,0.196}
\definecolor{commentgreen}{rgb}{0.247,0.5,0.3725}
\definecolor{darkpurple}{rgb}{0.5,0,0.33}

%% Some shortcuts for maths stuff
\providecommand{\Fm}{\mathtt{F}}
\providecommand{\Pm}{\mathtt{P}}
\providecommand{\Tp}{\textsuperscript{\textsf{T}}}
\providecommand{\xsub}[1]{\mathbf{x}_{#1}}
\providecommand{\X}{\mathbf{X}}
\providecommand{\Pmc}[2]{\mathbf{p}_{#1}^{#2\textsf{T}}}

%% TikZ command for line annotations
% Draw line annotation
% Input:
%   #1 Line offset (optional)
%   #2 Line angle
%   #3 Line length
%   #4 Line label
%   #5 Line colour
% Example:
%   \lineann[1]{30}{2}{$L_1$}
\newcommand{\lineann}[5][0.5]{%
    \begin{scope}[rotate=#2, #5,inner sep=2pt]
        \draw[dashed, #5!40] (0,0) -- +(0,#1)
            node [coordinate, near end] (a) {};
        \draw[dashed, #5!40] (#3,0) -- +(0,#1)
            node [coordinate, near end] (b) {};
        \draw[|<->|] (a) -- node[fill=white] {#4} (b);
    \end{scope}
}

%%%%%%%%%%%%%%%%%%%%%%%%%%%%%%%%%%%%%%%%%%%%%%%%%%%%%%%%%%%%%%%%%%%%%%%%%%%%%%%

\setcounter{page}{1}    % initialise page counter
\pagenumbering{arabic}  % use arabic numbers (change for index sections etc.?)

\ifdraft
 \setcounter{secnumdepth}{2}
\else
 \setcounter{secnumdepth}{2}
\fi

\setcounter{tocdepth}{2}

\setlength{\parindent}{0pt}
\setlength{\parskip}{6pt}

\lstset{    basicstyle=\small\ttfamily, 
            keywordstyle=\color{purple},
            commentstyle=\color{commentgreen}\textit,
            captionpos=b
        }



\begin{document}

%%%%%%%%%%%%%%%%%%%%%%%%%%%%%%%%%%%%%%%%%%%%%%%%%%%%%%%%%%%%%%%%%%%%%%%%%%%%%%%
% Title Page


\begin{flushright}
\authorname \\
% \authoremail 
\end{flushright}
\bigskip % have this so that vfill works to centre the title vertically

\vfill

\begin{center}
 \medskip
 {\Large\bf \projtitle}

 \vspace*{1cm}

 {\large Computer Science Tripos Part II}

 \bigskip

 {\large \authorcollege}

 \bigskip

 {\large \today}
\end{center}

\vfill
\vfill

\clearpage

%%%%%%%%%%%%%%%%%%%%%%%%%%%%%%%%%%%%%%%%%%%%%%%%%%%%%%%%%%%%%%%%%%%%%%%%%%%%%%%
% Proforma
\chapter*{Proforma}
{
\begin{tabular}{ll}
Name:               & \bf \authorname \\
College:            & \bf \authorcollege \\
Project Title:      & \bf \projtitle \\
Examination:        & \bf Computer Science Tripos, Part II, June 2011 \\
Word Count:         & \bf MANY words \\
Project Originator: & \bf Amir~H.~Hajizamani \\
Supervisor:         & \bf Cecilia Mascolo \\
\end{tabular}
}

\section*{Original Aims of the Project}

Lorem ipsum

\section*{Work Completed}

Lorem ipsum

\section*{Special Difficulties}

None.

\clearpage

%%%%%%%%%%%%%%%%%%%%%%%%%%%%%%%%%%%%%%%%%%%%%%%%%%%%%%%%%%%%%%%%%%%%%%%%%%%%%%%
% Declaration of Originality
\section*{Declaration}

I, \authorname~of \authorcollege, being a candidate for Part II of the Computer
Science Tripos, hereby declare that this dissertation and the work described in
it are my own work, unaided except as may be specified below, and that
the dissertation does not contain material that has already been used to any
substantial extent for a comparable purpose.

\bigskip
\leftline{Signed}

\medskip
\leftline{Date}

\vfill

\clearpage

%%%%%%%%%%%%%%%%%%%%%%%%%%%%%%%%%%%%%%%%%%%%%%%%%%%%%%%%%%%%%%%%%%%%%%%%%%%%%%%
% Table of Contents
\tableofcontents
\clearpage

%\clearpage
%\listoffigures

%\clearpage
%\listoftables

%\clearpage

%%%%%%%%%%%%%%%%%%%%%%%%%%%%%%%%%%%%%%%%%%%%%%%%%%%%%%%%%%%%%%%%%%%%%%%%%%%%%%%
% Introduction
\chapter{Introduction}

% - Mention fulfillment early
% - Principal Motivation
% - Other CS work

In this project I will explore and implement techniques for making 
recommendations to members of online social networks. The motivation behind 
this idea is that the value of data about an individual person or some digital 
content (text, audio, video, etc.) is increased by knowledge of its relation to 
other such entities. This is true for the user, whose experience is enriched, 
and the service provider, whose cultural and financial success will depend on 
behaviour-based analysis and the growth and interactions of its users.

\section{Motivation}

With the growing number of available social network services that individuals 
can choose to invest time and effort into, it is imperative that there is an 
incentive to do so. We take this investment by the user to consist of: 
providing personal information, creating and sharing content, making social 
connections with other users, and interacting with other users and content in 
the myriad of ways that are possible and observed in real world implementation 
such as Facebook, YouTube, LinkedIn, Twitter and Foursquare, to name a few. One 
way of providing this incentive is by making it easier for users to make 
\emph{discoveries} -- to find other individuals of interest (real-world 
friends, people with similar characteristics) or content. That is, make 
recommendations to users and keep them interested in exploring what the social 
network has to offer. Recommender systems aim to solve the problem of 
\emph{information overload} in areas where the users of a service struggle to 
make good choices on what content to consume or what social activities to take 
part in, simply because the options are overwhelming.

If we treat the users of a social network as a community, it is clear that the 
community will need the investment mentioned above from its users so that it 
reaches critical mass -- the point at which it is valuable for others to join 
or contribute to the network because the majority of their friends have, or so 
much of the content they consume is accessible via the service. This will 
continually increase the richness and quality of the service the community 
receives (assuming the service provider keeps up and adapts to changes!) and 
increase the lifetime and utility of the service for its users, too. Of course, 
here we are concerned with social networks that aim to reach the mainstream and 
are not inherently limited to a select number of users. For instance, a social 
network for the community of world experts on network congestion analysis is 
bound to hit its maximum membership limit very quickly and not be much more 
useful to its users than email or more traditional communication methods.

From a commercial perspective, which is arguably the more important one for the 
survival of a service, having users remain active on the network and 
incentivise others to do so, too, is paramount. It is well-understood that 
social network service providers are not expected to be profitable\footnote{The 
risks underlying the business of social networks -- 
\url{http://www.makeuseof.com/tag/how-do-social-networks-make-money-case-wondering/}} 
until they reach an ``audience of scale'' to enable the adoption of a sensible 
monetisation model. For instance, Facebook only became profitable in 2009\footnote{
\url{http://www.pcpro.co.uk/news/351646/facebook-eyes-profit-as-it-hits-300-million-members}} 
after reaching 300 million users. This means that the investors who enable such 
ventures at the start need to be convinced that the service has the potential 
to reach popularity and therefore provide a return on investment. Of the many 
strategies that can be used to maintain the growth of a social network, 
implementing a good recommender system is standard practice. It should be noted 
that recommendations are difficult, if not infeasible, to make at the very 
beginning of a network's lifetime due to the \emph{cold start} problem, that 
is, when there is insufficient data to base recommendations on. Therefore, the 
focus of this project is on stages of a network's lifetime after this issue has 
been overcome.

Below I describe my chosen dataset and the reasons for this choice over other 
available datasets.

\subsection*{The \emph{Mixcloud} Dataset}

I will work on data from Mixcloud.com whom I worked for over the summer of 
2009. They are a steadily-growing and successful content distribution website, 
aiming to be the ``YouTube of Radio''. They provide a platform for 
user-generated audio content, ranging from podcasts to DJ mixes, to be easily 
accessed by anyone. They have been operating for over a year and have in the 
order of 100,000 users and a similar number of so-called ``Cloudcasts'' 
(uploaded audio content). They provide their users with social networking 
features such as Twitter-style following, Facebook-style activity updates, 
favouriting of Cloudcasts and commenting, and a record of listening history. 
The user-uploaded Cloudcasts are also heavily annotated with tags, categories, 
text descriptions, tracklists where appropriate, metadata about recent 
listeners and more. 

This wealth of metadata about Mixcloud's content and users is available via 
their public API\footnote{\url{http://api.mixcloud.com}} and I shall be 
gathering the data I need from it. The fact that the API is public means that I 
am able to use the data in my project and the results will be publishable in 
the project dissertation. I have had further confirmation from the staff at 
Mixcloud Ltd. about this.

At the time of writing, Mixcloud implement some simple mechanisms for 
\emph{item recommendations} and \emph{personalised recommendations} -- 
recommending Cloudcasts similar to the current one, and recommending Cloudcasts 
matching the user's personal listening history. I would like my recommender 
system to primarily aim to provide the functionality of 
\emph{social recommendations} -- suggesting other users whom the current user 
may be interested in following or listening to based on their shared social 
connections, activity similarities and other metrics. The problem of 
recommending content to users based on the listening pattern of their own 
social connections (as opposed to that of their entire community) can also be 
called \emph{social recommendation} and would likely use similar underlying 
data and metrics. I should point out that the social aspect of the Mixcloud 
community is more similar to that of Twitter, where people follow users who 
produce good content as well as real life friends, than that of Facebook where 
social links are more often initiated physically.

Given the motivations for building recommender systems described in the last 
subsection, I understand that Mixcloud have prioritised the implementation of 
Cloudcast recommendations on their website because their main focus has been on 
building up the library of content on their network and promoting themselves as 
content distributors. However, the social networking layer on top of their 
distribution layer is naturally their next focus, whilst they improve the 
recommendation algorithms they currently use, too. They can leverage the high 
standard of their content metadata and existing social connections to make 
social recommendations. So essentially, the quality and completeness of the 
dataset, despite its organic nature, and the utility of a social recommender 
system for Mixcloud and their community are good reasons for picking it. I also 
compare the Mixcloud dataset with other ones traditionally used to explore 
recommender system techniques below.

A working system would likely draw on concepts from all three types of 
recommendation mentioned so far to account for inherent peculiarities in the 
dataset. For example, two users may have common taste in music but not be 
well-connected in the social graph or vice-versa. An effective system would 
take advantage of the rich social graph and indirect relationships between 
users and content to overcome these issues and enable further linkage in the 
dataset, hopefully with the effect of aiding the community and commercial 
objectives outlined earlier.

\subsubsection*{Comparison with other Available Datasets}

The most widely used datasets in discussions on recommender systems are the 
MovieLens, EachMovie, Book-Crossing and Jester Jokes datasets\footnote{
All of these datasets are available via\url{http://www.grouplens.org/node/12}} 
which contains items and user ratings on their respective titular item types. 
Though the quality of these datasets is proven, they have limitations for my 
intended project goals. In particular, there is no social connectivity data 
between users in these datasets and the links between entities are limited to 
ratings by users on some items. With the Mixcloud dataset I am able to explore 
the social graph of the users. Furthermore, the Mixcloud dataset provides much 
richer metadata about its entities, as described above, than any of the 
traditional datasets which give only enough data to build item-item and 
item-user matrices corresponding to the ratings. The Mixcloud dataset has scope 
for more interesting analysis and recommender system implementations.

The Mixcloud dataset also has the potential for analysis in the temporal 
dimension, which is the extension idea for this project and is described later.

\subsection*{The Temporal Dimension}

\section{Wider Context}

The Web is rife with successful recommender systems, with companies such as 
\textit{Amazon, Netflix\footnote{In 2006 Netflix challenged programmers to 
improve their movie recommendation engine by 10\%.}} and \textit{Pandora} 
(radio service backed by \textit{The Music Genome Project}\footnote{The MGP 
uses the``genealogy'' of music to link songs and artists -- 
\url{http://www.pandora.com}}) investing a lot of time and money in improving 
their algorithms and data quality to better recommend items of interest to 
their users. On the other hand, more directly community-driven websites such as 
\textit{del.icio.us} and \textit{flickr} have harnessed the collective 
intelligence of their users and their tagging of their content with keywords. 
Collaborative tagging leads to an organic categorisation and annotation of data 
which has been termed \textit{folksonomy}\footnote{A portmanteau of 
\textit{folk} and \textit{taxonomy} -- 
\url{http://www.vanderwal.net/folksonomy.html}}. In a sufficiently large and 
active community, the folksonomy that can be extracted becomes a good source of 
input for the algorithms in a recommender system. 

With the accelerating growth of social networks such as Twitter and Facebook, 
the value of social recommendations which aim to increase the connections 
between individuals is more important that it used to be: the items of 
interests are now other users not products on sale (though from an advertising 
revenue perspective the difference is blurred). Recommender systems are nearing 
maturity and will soon be, if they are not already, commodity technologies as 
the ``social connections layer'' of virtual communities is completed\footnote{
The ``game layer'' is the next stage in building virtual communities, the 
founder of SCVNGR argues -- \url{
http://www.ted.com/talks/seth_priebatsch_the_game_layer_on_top_of_the_world.html}}. 
This means that an analysis of such systems is essential to the understanding 
of social networking and the Web as it experiences a paradigm shift from a 
\emph{search} platform (e.g. Google) to a \emph{discovery} platform.


%%%%%%%%%%%%%%%%%%%%%%%%%%%%%%%%%%%%%%%%%%%%%%%%%%%%%%%%%%%%%%%%%%%%%%%%%%%%%%%
% Preparation 
\chapter{Preparation}
% - before coding
% - Refinement of the proposal
% - Professionalilty
% - SoftEng practices

\section{Current Solutions}

\section{Literature Review}

\section{Deciding on an Approach}

\section{Tools}

 \subsection{Programming Languages}

 \subsection{Databases}

%%%%%%%%%%%%%%%%%%%%%%%%%%%%%%%%%%%%%%%%%%%%%%%%%%%%%%%%%%%%%%%%%%%%%%%%%%%%%%%
% Implementation
\chapter{Implementation}
% - Code produced
% - Acknowledge use of existing tools
% - Major milestone

\section{The Dataset}

\section{The Recommender}

%%%%%%%%%%%%%%%%%%%%%%%%%%%%%%%%%%%%%%%%%%%%%%%%%%%%%%%%%%%%%%%%%%%%%%%%%%%%%%%
% Evaluation
\chapter{Evaluation}
% - Signs of success
% -  thorough and systematic testing
% - figures
% - Goals achieved
% - Short-comings
% - Amibtious logical conclusion

\section{Measures}

\section{Static}

\section{Temporal}

%%%%%%%%%%%%%%%%%%%%%%%%%%%%%%%%%%%%%%%%%%%%%%%%%%%%%%%%%%%%%%%%%%%%%%%%%%%%%%%
% Conclusions
\chapter{Conclusions}
% - Reminder of aims and achievements
% - Discussion in hindsight

\section{Further work}

%%%%%%%%%%%%%%%%%%%%%%%%%%%%%%%%%%%%%%%%%%%%%%%%%%%%%%%%%%%%%%%%%%%%%%%%%%%%%%%
% Bibliography

%%%%%%%%%%%%%%%%%%%%%%%%%%%%%%%%%%%%%%%%%%%%%%%%%%%%%%%%%%%%%%%%%%%%%%%%%%%%%%%
% Appendices

\appendix
\chapter*{Appendix}
\addcontentsline{toc}{chapter}{Appendix}

\section{First Appendix}

\end{document}
