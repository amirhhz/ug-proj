%%%%%%%%%%%%%%%%%%%%%%%%%%%%%%%%%%%%%%%%%%%%%%%%%%%%%%%%%%%%%%%%%%%%%%%%%%%%%%%
% PART II PROJECT DISSERTATION - RECOMMENDER SYSTEMS
%
% Amir H. Hajizamani, ahh29@cam.ac.uk
%
%%%%%%%%%%%%%%%%%%%%%%%%%%%%%%%%%%%%%%%%%%%%%%%%%%%%%%%%%%%%%%%%%%%%%%%%%%%%%%%

\documentclass[a4paper,12pt,twoside,notitlepage]{report}

\author{Amir H. Hajizamani}
\title{CST Part II Individual Project Dissertation - Recommender Systems for 
Social Networks}

\usepackage{verbatim}
\usepackage[T1]{fontenc}
\usepackage{a4wide}
\usepackage{parskip}
%\usepackage{hyperref}
\usepackage{url}
\usepackage{amsmath}
%\usepackage{proof}
\usepackage{comment}

% Conditionals (for draft mode)
\usepackage{ifthen}

% Headers
\usepackage{fancyhdr}

% TOC customisation
\usepackage{tocloft}

% Captions
\usepackage{subfig}
\usepackage{caption}

% Figures
\usepackage{wrapfig}
\usepackage{graphicx}
\usepackage{tikz}
\usetikzlibrary{positioning,mindmap,chains,shapes.misc,arrows,fit}
\usetikzlibrary{decorations.pathmorphing}

% Tables
\usepackage{booktabs}
\usepackage{colortbl}
\usepackage{longtable}

% Palatino font - remove to go back to normal CMR
\usepackage{mathpazo}

% Colour
\usepackage{color}

% Listings (for code snippets ..._
\usepackage{listings}

% Algorithms
\usepackage[boxed]{algorithm}

%\makeindex

\frenchspacing

%%%%%%%%%%%%%%%%%%%%%%%%%%%%%%%%%%%%%%%%%%%%%%%%%%%%%%%%%%%%%%%%%%%%%%%%%%%%%%%
% The BIG DRAFT SWITCH - Set to true to enable draft mode
%
% !!! This will produce extra output that 
%     is not part of the final dissertation !!!

\newboolean{draft}
\setboolean{draft}{true}

%%%%%%%%%%%%%%%%%%%%%%%%%%%%%%%%%%%%%%%%%%%%%%%%%%%%%%%%%%%%%%%%%%%%%%%%%%%%%%%

%%%%%%%%%%%%%%%%%%%%%%%%%%%%%%%%%%%%%%%%%%%%%%%%%%%%%%%%%%%%%%%%%%%%%%%%%%%%%%%
% Definitions

\def\projtitle{Recommender Systems for Social Networks}
\def\authorname{Amir~H.~Hajizamani} 
\def\authoremail{\url{ahh29@cam.ac.uk}}
\def\authorcollege{St.~John's~College}
\def\projsupervisor{Cecilia~Mascolo}

\def\mixurl{\emph{mixcloud.com}}


%%%%%%%%%%%%%%%%%%%%%%%%%%%%%%%%%%%%%%%%%%%%%%%%%%%%%%%%%%%%%%%%%%%%%%%%%%%%%%%
% Custom Environments

\newcommand{\fixme}[1]{\ifdraft{\color{red} {\bf Fixme: \sc #1}}\fi}
\newcommand{\todo}[1]{\ifdraft{\textsf{\color{red} TODO: #1}}\fi}
\newcommand{\remark}[1]{\ifdraft{\textsf{\color{blue} #1}}\fi}

%% Colours
\definecolor{darkgreen}{rgb}{0,0.3137,0.196}
\definecolor{commentgreen}{rgb}{0.247,0.5,0.3725}
\definecolor{darkpurple}{rgb}{0.5,0,0.33}

%% Some shortcuts for maths stuff
\providecommand{\Fm}{\mathtt{F}}
\providecommand{\Pm}{\mathtt{P}}
\providecommand{\Tp}{\textsuperscript{\textsf{T}}}
\providecommand{\xsub}[1]{\mathbf{x}_{#1}}
\providecommand{\X}{\mathbf{X}}
\providecommand{\Pmc}[2]{\mathbf{p}_{#1}^{#2\textsf{T}}}

%% TikZ command for line annotations
% Draw line annotation
% Input:
%   #1 Line offset (optional)
%   #2 Line angle
%   #3 Line length
%   #4 Line label
%   #5 Line colour
% Example:
%   \lineann[1]{30}{2}{$L_1$}
\newcommand{\lineann}[5][0.5]{%
    \begin{scope}[rotate=#2, #5,inner sep=2pt]
        \draw[dashed, #5!40] (0,0) -- +(0,#1)
            node [coordinate, near end] (a) {};
        \draw[dashed, #5!40] (#3,0) -- +(0,#1)
            node [coordinate, near end] (b) {};
        \draw[|<->|] (a) -- node[fill=white] {#4} (b);
    \end{scope}
}

%%%%%%%%%%%%%%%%%%%%%%%%%%%%%%%%%%%%%%%%%%%%%%%%%%%%%%%%%%%%%%%%%%%%%%%%%%%%%%%

\setcounter{page}{1}    % initialise page counter

% pre-TOC page formatting
\pagenumbering{roman}
\pagestyle{plain}

%\setcounter{secnumdepth}{2}
%\setcounter{tocdepth}{2}

\setlength{\parindent}{0pt}
\setlength{\parskip}{6pt}
\linespread{1.2}

\begin{document}

%%%%%%%%%%%%%%%%%%%%%%%%%%%%%%%%%%%%%%%%%%%%%%%%%%%%%%%%%%%%%%%%%%%%%%%%%%%%%%%
% Title Page

\thispagestyle{empty} 

\begin{flushright}
\authorname
\end{flushright}
\bigskip % have this so that vfill works to centre the title vertically

\vfill

\begin{center}
 \medskip
 {\Large\bf \projtitle}

 \vspace*{1cm}

 {\large Computer Science Tripos Part II}

 \bigskip

 {\large \authorcollege}

 \bigskip

 {\large \today}
\end{center}

\vfill

\clearpage


%%%%%%%%%%%%%%%%%%%%%%%%%%%%%%%%%%%%%%%%%%%%%%%%%%%%%%%%%%%%%%%%%%%%%%%%%%%%%%%
% Proforma
\section*{Proforma}

\begin{tabular}{ll}
Name:               & \bf \authorname \\
College:            & \bf \authorcollege \\
Project Title:      & \bf \projtitle \\
Examination:        & \bf Computer Science Tripos, Part II, June 2011 \\
Word Count:         & \bf \todo{MANY words} \\
Project Originator: & \bf \authorname \\
Supervisor:         & \bf \projsupervisor \\
\end{tabular}

%\footnotetext[1]{This word count was computed
%by {\tt detex diss.tex | tr -cd '0-9A-Za-z $\tt\backslash$n' | wc -w}}

\subsection*{Original Aims of the Project}

To build a system that would recommend new social links to users of a social
network in order to increase the connectivity of its social graph. These
recommendations would be based on the current state of the graph and their
quality would depend on whether the recommended links actually occur in a
future state of the graph. From the outset, the dataset to be used for the
project was that of the online audio-sharing community \mixurl.

\subsection*{Work Completed}

The initial part of the work was obtaining the Mixcloud dataset. I modelled the
dataset's entities and wrote an abstraction layer for the Mixcloud API. I
then used this in a crawler program that downloaded and stored as much of the
data on \mixurl\ as possible.

The subsequent recommender system I wrote achieves the above aims by modelling
the recommendation process as an edge prediction problem: given a social graph,
it tries to predict what edges will form in the future between unconnected nodes
and assigns a confidence value to the predictions. In essence, this is done by
calculating similarity values between nodes and choosing the most similar
unconnected nodes as predictions.

\subsection*{Special Difficulties}

None.

\newpage

%%%%%%%%%%%%%%%%%%%%%%%%%%%%%%%%%%%%%%%%%%%%%%%%%%%%%%%%%%%%%%%%%%%%%%%%%%%%%%%
% Declaration of Originality
\section*{Declaration}

I, \authorname~of \authorcollege, being a candidate for Part II of the Computer
Science Tripos, hereby declare that this dissertation and the work described in
it are my own work, unaided except as may be specified below, and that
the dissertation does not contain material that has already been used to any
substantial extent for a comparable purpose.

\bigskip
\bigskip
\bigskip
\leftline{Signed}

\medskip
\leftline{Date}


\newpage

%%%%%%%%%%%%%%%%%%%%%%%%%%%%%%%%%%%%%%%%%%%%%%%%%%%%%%%%%%%%%%%%%%%%%%%%%%%%%%%
% Acknowledgements

\section*{Acknowledgements}

If this dissertation turns out to be anything worth the paper it's printed on I
have the following people to thank:

\begin{itemize}
 \item {\bf Liam~McNamara}
 \item {\bf Helen~Ennos}, for all her feedback, encouragement and support.
Without you, my time at Cambridge would have been a very different experience. 
\end{itemize}

\clearpage

%%%%%%%%%%%%%%%%%%%%%%%%%%%%%%%%%%%%%%%%%%%%%%%%%%%%%%%%%%%%%%%%%%%%%%%%%%%%%%%
% Table of Contents
\tableofcontents
\clearpage

%\clearpage
%\listoffigures

%\clearpage
%\listoftables

%\clearpage


%%%%%%%%%%%%%%%%%%%%%%%%%%%%%%%%%%%%%%%%%%%%%%%%%%%%%%%%%%%%%%%%%%%%%%%%%%%%%%%

% Switch to arabic page numbers after TOC
\setcounter{page}{1}
\setcounter{chapter}{0}
\pagenumbering{arabic}
\pagestyle{headings}


%%%%%%%%%%%%%%%%%%%%%%%%%%%%%%%%%%%%%%%%%%%%%%%%%%%%%%%%%%%%%%%%%%%%%%%%%%%%%%%
% Introduction
\chapter{Introduction}

% - Mention fulfillment early
% - Principal Motivation
% - Other CS work

In this chapter I will explain my motivations for investigating recommender
systems in general and the specific focus points of my project as
presented in this dissertation using data from \mixurl.

\section{Motivation}

The basic concept that underpins recommender systems is that the value of data
about an individual person or piece of content is increased by knowledge of its
relation to other such entities. This is true for the user of a system, whose
experience is enriched with this metadata and is thus incentivised to keep
using, it and the service provider, whose cultural impact and financial success
will depend the growth and interactions of its users with the service.

There are now a growing number of available social networking services that
individuals can choose to invest time and effort into. We take this investment
by the user to consist of: providing personal information, creating and sharing
content, making social connections with other users, and interacting with other
users and content. One way of providing an incentive to do this is by making it
easier for users to make \emph{discoveries} -- to find other individuals or
items of interest (real-world friends, content that compliments the user's
favourites). That is, make recommendations to users and keep them interested in
exploring what the social network has to offer. Recommender systems aim
to solve the problem of \emph{information overload} in areas where the users of
a service struggle to make good choices on what content to consume or what
social activities to take part in, simply because the options are overwhelming.

If we treat the users of a social network as a community, it is clear that such
a community will need the investment mentioned above from its users to reach
some critical mass -- the point at which it is valuable for others to join 
or contribute to the network because the majority of their friends have, or so 
much of the content they consume is accessible via the service. This will 
continually increase the richness and quality of the service the community 
receives (assuming the service provider keeps up and adapts to changes!) and 
increase the lifetime and utility of the service for its users, too. Of course, 
here we are concerned with social networks that aim to reach the mainstream and 
are not inherently limited to a select number of users. For instance, a social 
network for the community of world experts on network congestion analysis is 
bound to hit its maximum membership limit very quickly and not be much more 
useful to its users than email or more traditional communication methods.

From a commercial perspective of a service provider, having users remain active
on the network and incentivise others to do so, too, is paramount. It is
well-understood that social network service providers are not expected to be
profitable\footnote{The 
risks underlying the business of social networks -- 
\url{http://www.makeuseof.com/tag/how-do-social-networks-make-money-case-wondering/}} 
until they reach an ``audience of scale'' to enable the adoption of a sensible 
monetisation model. For instance, Facebook only became profitable in 2009\footnote{
\url{http://www.pcpro.co.uk/news/351646/facebook-eyes-profit-as-it-hits-300-million-members}} 
after reaching 300 million users. This means that the investors who enable such 
ventures at the start need to be convinced that the service has the potential 
to reach popularity and therefore provide a return on investment. Of the many 
strategies that can be used to maintain the growth of a social network, 
implementing a good recommender system is standard practice. It should be noted 
that recommendations are difficult, if not infeasible, to make at the very 
beginning of a network's lifetime due to the \emph{cold start} problem, that 
is, when there is insufficient data to base recommendations on. Therefore, the 
focus of this project is on stages of a network's lifetime when it is
nearinafterg maturity and this issue hasbeen overcome.

\subsection{Wider Context}

The Web is rife with successful recommender systems, with companies such as 
\emph{Amazon, Netflix\footnote{In 2006 Netflix challenged programmers to 
improve their movie recommendation engine by 10\%.}} and \emph{Pandora} 
(radio service backed by \emph{The Music Genome Project}\footnote{The MGP 
uses the``genealogy'' of music to link songs and artists -- 
\url{http://www.pandora.com}}) investing a lot of time and money in improving 
their algorithms and data quality to better recommend items of interest to 
their users. On the other hand, more directly community-driven websites such as 
\emph{del.icio.us} and \emph{flickr} have harnessed the collective 
intelligence of their users and their tagging of their content with keywords. 
Collaborative tagging leads to an organic categorisation and annotation of data 
which has been termed \emph{folksonomy}\footnote{A portmanteau of 
\emph{folk} and \emph{taxonomy} -- 
\url{http://www.vanderwal.net/folksonomy.html}}. In a sufficiently large and 
active community, the folksonomy that can be extracted becomes a good source of 
input for the algorithms in a recommender system. 

With the accelerating growth of social networks such as Twitter and Facebook, 
the value of social recommendations which aim to increase the connections 
between individuals is more important that it used to be: the items of 
interests are now other users not products on sale (though from an advertising 
revenue perspective the difference is blurred). Recommender systems are nearing 
maturity and will soon be, if they are not already, commodity technologies as 
the ``social connections layer'' of virtual communities is completed\footnote{
The ``game layer'' is the next stage in building virtual communities, the 
founder of SCVNGR argues -- \url{
http://www.ted.com/talks/seth_priebatsch_the_game_layer_on_top_of_the_world.html
}}. 
This means that an analysis of such systems is essential to the understanding 
of social networking and the Web as it experiences a paradigm shift from a 
\emph{search} platform (e.g. Google) to a \emph{discovery} platform.


\section{The \mixurl\ Dataset}

I will work on data from the \mixurl. They are a successful audio content
distribution website, aiming to be the ``YouTube of Radio''. They provide a
platform for user-generated audio content, ranging from podcasts to DJ mixes, to
be easily accessed by anyone. They have been operating for nearly two years and
have in the order of 100,000 users and a similar number of
so-called ``Cloudcasts'' (uploaded audio content). They provide
their niche community of users with social networking features such as
Twitter-style following, Facebook-style activity updates, favouriting of
Cloudcasts and commenting, and a record of listening history. 
The user-uploaded Cloudcasts are also heavily annotated with tags, categories, 
text descriptions and other metadata.This wealth of metadata about Mixcloud's
users and content is available via their
public API\footnote{\url{http://api.mixcloud.com}} and I gathered the data I
need from it.

\subsection{Focus on Social Recommendations}
At the time of writing, \mixurl\ implements some simple mechanisms for 
\emph{item recommendations} and \emph{personalised recommendations}: 
respectively, recommending Cloudcasts similar to the one being viewed/played,
and recommending Cloudcasts related to the user's personal listening history. 

My approach is to provide \emph{social recommendations}: suggesting \emph{other
users} whom the current user may be interested in following or listening to. My
reason for this is that the social aspect of the \mixurl\ community is more
similar to that of Twitter, where people connect with users who produce good
content as well as real life friends, than that of Facebook where social links
are more often initiated physically, or in the traditional sense of ``social''.

Given the motivations for building recommender systems described above, I
understand that the Mixcloud team have prioritised the implementation of 
the recommendations centred around content because their main
focus has been on building up the library of content on their network and
promoting themselves as content distributors. However, the social networking
layer on top of their distribution layer is the next natural focus. 

\subsection{The Temporal Dimension}

As I am collecting my own data firsthand, I can look at the performance of my
recommendations over temporal snapshots of the dataset.
\todo{more detail here?}
\subsection{Comparison with other Available Datasets}

The most widely used datasets in discussions on recommender systems are the 
MovieLens, EachMovie, Book-Crossing and Jester Jokes datasets\footnote{
All of these datasets are available via\url{http://www.grouplens.org/node/12}} 
which contain items and user ratings on their respective titular item types. 
Though the quality of these datasets is proven, they have limitations for my 
intended project goals. In particular, there is no social connectivity data 
between users in these datasets and the links between entities are limited to 
ratings by users on some items. With the Mixcloud dataset I am able to explore 
the social graph of the users. Furthermore, the Mixcloud dataset provides much 
richer metadata about its entities, as described above, than any of these 
traditional datasets which give only enough data to make item
recommendations. The Mixcloud dataset has scope for more interesting analysis.

%%%%%%%%%%%%%%%%%%%%%%%%%%%%%%%%%%%%%%%%%%%%%%%%%%%%%%%%%%%%%%%%%%%%%%%%%%%%%%%
% Preparation 
\chapter{Preparation}
\begin{verbatim}
% - before coding
% - Refinement of the proposal
% - Professionalilty
% - SoftEng practices 
\end{verbatim}

\section{The Anatomy of the \mixurl\ Dataset}

\begin{verbatim}
 - table of mixcloud API resources
 - text description
\end{verbatim}

\section{Programming Languages and New Tools}

\subsection{Python}

\begin{verbatim}
- Interpreted, high level -> faster development
- Many existing libraries 
\end{verbatim}


\subsection{Non-Relational Databases}

\begin{verbatim}
- JSON-based Document-oriented DBs: MongoDB, CouchDB
- in-memory key-value stores: Redis, Memcached
\end{verbatim}

\section{Project Management}

\begin{verbatim}
- Some crap about "Iterative Model" of SoftEng ... ?! 
\end{verbatim}

\subsection{Version Control and Backup Strategy}
\begin{verbatim}
- Subversion on SRCF, daily automated backups etc.
- Dataset gathered and stored in 3 locations 
\end{verbatim}

\subsection{Development Environment}
\begin{verbatim}
- PyCharm IDE (svn integration, cross-platform)
\end{verbatim}


%%%%%%%%%%%%%%%%%%%%%%%%%%%%%%%%%%%%%%%%%%%%%%%%%%%%%%%%%%%%%%%%%%%%%%%%%%%%%%%
% Implementation
\chapter{Implementation}

\begin{verbatim}
% - Code produced
% - Acknowledge use of existing tools
% - Major milestone 
\end{verbatim}


\section{Gathering the Data}

\subsection{API Abstraction: A Python Wrapper}

\begin{verbatim}
- modelled on similar Facebook API Wrapper
- Thorough object-oriented view representing anatomy as above
- paged data complication
- proxying, caching
\end{verbatim}

\subsection{The Crawler}

\begin{verbatim}
- proxying, caching
- breadth-first search of graph using features
- keep track of state with Redis 
\end{verbatim}

\subsection{Storing the Data}

\subsubsection{MongoDB: A Document Store}
\begin{verbatim}
- JSON response directly to MongoDB
- Complication with async writes
- Complication with document and db size limit
\end{verbatim}

\section{Cleaning the Data}

\begin{verbatim}
- API bugs
- Temporal errors (asymmetry)
\end{verbatim}

\subsection{Dataset Store Abstraction: MongoMix}
\begin{verbatim}
- MongoMix, abstraction of Mixcloud Data in MongoDB collections
- Cleaning methods: count correction, asymmetry fixes
\end{verbatim}

\section{Exploring the Data}

\subsection{Available Features}
\begin{verbatim}
- relevant features
- stats on the data
\end{verbatim}

\subsection{Producing Test Datasets}
\begin{verbatim}
- "hiding" edges
\end{verbatim}

\section{Making Recommendations}

\subsection{Social Graph Model and Edge Prediction}

\subsection{Similarity Measures}

\subsubsection{Intersection Size}

\subsubsection{Jaccard Co-efficient}

\subsubsection{Modified Jaccard}

\subsubsection{Adamic-Adar}

\subsubsection{Preferential Attachment}

%%%%%%%%%%%%%%%%%%%%%%%%%%%%%%%%%%%%%%%%%%%%%%%%%%%%%%%%%%%%%%%%%%%%%%%%%%%%%%%
% Evaluation
\chapter{Evaluation}

\begin{verbatim}
% - Signs of success
% -  thorough and systematic testing
% - figures
% - Goals achieved
% - Short-comings
% - Amibtious logical conclusion 
\end{verbatim}

\section{Static}

\section{Temporal}

%%%%%%%%%%%%%%%%%%%%%%%%%%%%%%%%%%%%%%%%%%%%%%%%%%%%%%%%%%%%%%%%%%%%%%%%%%%%%%%
% Conclusions
\chapter{Conclusions}
% - Reminder of aims and achievements
% - Discussion in hindsight

\section{Further work}

%%%%%%%%%%%%%%%%%%%%%%%%%%%%%%%%%%%%%%%%%%%%%%%%%%%%%%%%%%%%%%%%%%%%%%%%%%%%%%%
% Bibliography

%%%%%%%%%%%%%%%%%%%%%%%%%%%%%%%%%%%%%%%%%%%%%%%%%%%%%%%%%%%%%%%%%%%%%%%%%%%%%%%
% Appendices

\appendix
\chapter*{Appendix}
\addcontentsline{toc}{chapter}{Appendix}

\section{First Appendix}

\end{document}
